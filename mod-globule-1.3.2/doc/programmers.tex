\section{Building a distribution}

No package file should exists, as ``make dist'' silently refuses to overwrite
it

\begin{verbatim}
make clena
make cleanest
make ;# all configure steps using GNUmakefile
rm mod-globule-1.3.0.tar.gz ; make dist
cp mod-globule-1.3.0.tar.gz /home/ftp/pub/globule/auto/src
./tools/mkinstaller.sh
\end{verbatim}

\section{Globule demo environment}

If you have a fairly recent and complete Unix/Linux environment, without
any special needs for your web-server nor any pre-installed paths where
documents should go, but you do want to compile everything yourself, then
the following procedure helps you out.

This procedure builds a complete Globule system, including Apache server,
PHP, and other tools is a single location.  When you have given Globule
a try, but want to clean up for some reason, you can just delete the
directory tree.  Preferably this end-location should be /usr/local/globule,
but any location that is not already in use will do.

The procedure is especially helpful to build demo-systems.  By downloading
just a single shell script (a shar archive), you get all the software.


\section{Vrije Universiteit specific notes}

For the Solaris systems at the VU you need to use:

  ./installer.sh --extra-php-config="--with-libxml-dir=/home/berry --with-curl=/home/berry --without-gd"

The webalizer support and some monitoring will not work due to the missing GD
library.
You really want to use the flag --keep-build too and when upgrading the flag
--install-demo can be used to re-install the demo globulectl if you 


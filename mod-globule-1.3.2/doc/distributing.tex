\documentclass[10pt,twoside]{article}

\usepackage{a4}
\usepackage[obeyspaces]{url}
\usepackage{graphicx}
\usepackage{hevea}
\usepackage{fancyvrb}

\def\section{\@startsection{section}{1}
{\z@}{-2.5ex plus -0.5ex minus -0.1ex}{0.5ex plus 0.1ex}{\large\bf}}
\def\subsection{\@startsection{subsection}{2}
{\z@}{-2.25ex plus -0.3ex minus -0.2ex}{0.05ex plus 0.05ex}{\normalsize\bf}}
\def\subsubsection{\@startsection{subsubsection}{3}
{\z@}{-2.25ex plus -0.3ex minus -0.2ex}{0.05ex plus 0.05ex}{\normalsize\sc}}
\def\paragraph{\@startsection{paragraph}{4}
{\z@}{2ex plus 0.5ex minus 0.1ex}{-0.7em}{\normalsize\it}}

\def\@listi{\topsep 0.4ex \parsep 0pt \itemsep \parsep}
\topsep 0.4ex \partopsep 0pt \itemsep 0pt \parsep 0pt
\def\@listii{\leftmargin\leftmarginii
 \labelwidth\leftmarginii\advance\labelwidth-\labelsep
 \topsep 0pt \parsep 0pt \itemsep \parsep}
\def\@listiii{\leftmargin\leftmarginiii
 \labelwidth\leftmarginiii\advance\labelwidth-\labelsep
 \parsep \z@ \topsep 0pt \partopsep 0pt \itemsep \topsep}

\setcounter{tocdepth}{3}
%\setcounter{cuttingdepth}{3}
\ifhevea
\newcommand{\l}{l}
\newcommand{\vfill}{}
\tocnumber
\htmlprefix{Globule User Manual: }
\newenvironment{p}{}{\par}
\clubpenalty=10000
\widowpenalty=10000
\fi

\title{Globule Packaging Guide}
\date{Version 1.3.0
  \ifhevea \begin{rawhtml}</H3> 
<H3 ALIGN=center>\end{rawhtml} \else \\ \fi
  \ahrefurl{http://www.globule.org/}}
\author{\ahref{\url{http://www.halderen.net/}}{Berry~van~Halderen}}

\begin{document}
\maketitle

\tableofcontents
\cleardoublepage

\parindent 0pt
\parskip 1em

\section{Introduction}

Packaging involves preparing the Globule software ready for release.  This
guide should describe all steps in the final approach for a release of
Globule, where to retrieve the installation from, how to build the packages,
etcetera.

This guide is based on Globule version 1.3.0.  It need not be necessarily
out-of-date for coming releases.  The Globule Broker System (GBS) is not part
of the Globule distribution, but is maintained seperately.

\begin{quote}\scriptsize
For reference purposes: \\
The Globule Broker System (GBS) is located in CVS repository available through
\newline
\verb!:ext:flits.cs.vu.nl:/net/globe/CVSRepos2! under the module name
\texttt{globeworld}.  It has a similar, albeit simpler
description in the file \texttt{README.dist} in the main
directory of the CVS module.  GBS and Globule releases are unrelated at this
time.
\end{quote}

The packaging guide should first the studied fully, before making a
distribution, as some consequences of steps are later explained.  The guide
should however be a cronilogical step-by-step walkthrough that can be followed
to make a distribution.  In fact, all steps are mostly required unless you
know what you are doing, as some steps in building one type of package are
sometimes needed in building another.

\subsection{Updating this document}

To keep this document up to date regulary review the procedure.  If additions
or changes need to be made, \emph{also update the version number} in this
document to reflect that this guide has been updated to the associated version
number in Globule.

When making important changes which will not work on previous releases of
Globule, make a historic note in the appendix.  This should only be done when
a procedure conflicts with earlier distribution, not with additions.

\section{Repository and packages}

The Globule source code is kept in a CVS repository available at CVSROOT
\verb!:ext:flits.cs.vu.nl:/net/globe/CVSRepos2! with module name
\texttt{globule}.

\section{System setup}

The procedure for building the distribution packages assumes a number of paths
where to build the packages.  These packages are not necessarily fixed, but
some scripts (e.g. \verb!mkinstaller.sh!) should be modified if different
paths are to be used.  All packages are assumed to be build on the same
machine, but VMware is used to build a number of the packages in the specific
Linux distributions.  Images for the virtual machines to be used with VMware
are to be supplied (see appendix~\ref{sec:vmware}).

The following computer setup is assumed:
\begin{enumerate}
\item Some linux box (used: Slackware-10.1).

\item RPMs tree in \verb!/usr/src/rpm!.  This is however \verb!/usr/src/redhat!
under the % mighty irritating
RedHat distributions.

Replace \verb!/usr/src/rpm! by
\verb!/usr/src/redhat! and edit the files \verb!GNUmakefile!,
\verb!publish.sh! and verb!tools/mkinstaller.sh! to reflect changes if you
need to.

\item The final distribution is produced in /home/ftp/pub/globule.

It is usefull if this tree is also the source for the installer.sh script
auto/ directory for updates.

\item The path to apache is assumed to be \verb!../globule! when inside the Globule source directory. This used in the \texttt{GNUmakefile}.

\item The host \verb!flits.cs.vu.nl! is assumed to run a ntp daemon.

\end{enumerate}

Try to use the GNUmakefile in the distribution.  This makefile, which works as
a front of the real makefile when using GNU make, adds a number of tasks
especially usefull for the maintainer of Globule:

\begin{itemize}
\item it makes sure the autoconf, automake, etcetera are run properly and
produced files are up to date;
\item it calls configure when needed.  The options to configure and the path
to apache are however hardcoded and may need changing.
\item it provides checks to make sure the version number of Globule is the
same in a number of files.
\item the make target \texttt{cleanest} should remove
all additional files which should never go into the CVS repository.
\item it will run make more silently for the maintainer only, so that the
standard makefile to end-users is normally verbose.
\end{itemize}

\section{Sanity check of prepared distribution}

Before actually making the distribution you should take the time to do
a sanity check of the files to be distributed.

\subsection{Version check}

\begin{itemize}
\item Check that the version in the file \texttt{VERSION} is
right.

\item To check that other files which contain the version number of Globule
are also up-to-date, run using the GNUmakefile:
\begin{verbatim}
  make versioncheck
\end{verbatim}
This should produce no errors, if not, check the related files all use
the same version for Globule.

\item Check that the version number in the documentation is up-to-date

\textbf{missing explanation}

\item Check the distribution release files \texttt{ChangeLog} and \texttt{NEWS}.

\textbf{missing explanation}

\item Run a clean compilation using the command sequence:
\begin{verbatim}
  make clean clean cleanest
  make all
\end{verbatim}
This should still compile cleanly (procedure assumes using GNUmakefile).

\item Test using the test suite
\begin{verbatim}
  cd testsuite ; ./runtests.sh ; cd ..
\end{verbatim}
Test backuptest.pl isn't working properly, ignore it.
Test uritest.pl is suppost to fail.  Test redirectortest.pl reports as
failed, but inspection of the *.out file should make it clear that only
unexpected re-re-redirects are being made.

\item Clean up previous steps using
\begin{verbatim}
  make clean cleanest
\end{verbatim}
Then check that all files are up-to-date against the CVS repository.
Make sure all sources are checked in.

\end{itemize}

\section{Creating the packages}

This assumes you have \emph{really} done the steps in the sanity section.  If
not, the distribution packages may include wrong files and/or outdated.  The
following procedure namely assumes a clean tree which is totally equal to the
CVS repository.

\begin{itemize}

\item Tag the CVS repository with a tag naming the distribution version using
\begin{verbatim}
cvs tag \textit{tagname}
\end{verbatim}

\item Administrate the tag in
\verb!nfs://flits.cs.vu.nl/net/globe/CVSRepos2.Tags.globule!.  Also look at
this files for possible naming conventions.

\end{itemize}

\subsection{Main distribution file}

\begin{itemize}

\item Now produce the dist file
\begin{verbatim}
  make configure
  rm mod-globule-*.tar.gz
  make dist
\end{verbatim}
Not removing any dist file with the same version will silently refuse to
reproduce the dist file.

\item Make sure the SPEC files are up-to-date.

\textbf{Explaination missing, and possibly better move to another section}

\item Upgrading apache/httpd and others external packages

\textbf{Explaination missing, and possibly better move to another section}

\item Copy the \verb!mod-globule-*.tar.gz! to \verb!/usr/src/rpm!
(required location for \textbf{missing}).

\end{itemize}

\subsection{Building the Windows Release}

\subsubsection{Prerequisites}

\begin{itemize}
\item[-] This build description assumes Microsoft Visual C++ version 7.1.3088,
of course, MicroSoft has a nasty habit of changing their interface constantly,
so start exhaustively searching your menu's to change the parameters.
\item[-] Nullsoft installer builder.
\item[-] Utility programs \verb!patch! and \verb!awk!.
\item[-] Files \verb!msvcp71.dll!, \verb!msvcr71.dll!, \verb!msvcp71d.dll!,
and\verb!msvcr71d.dll!.
\end{itemize}

\subsubsection{Preparation}

\begin{enumerate}
\item Create directories \verb!C:\ApacheInstallFiles! and
\verb!C:\ApacheInstallFiles\PreInstalled!.

\item Extract source code for Globule in \verb!C:\globule!
I.e. do a cvs checkout of the globule module while in \verb!C:\!

\item Download Apache source \verb!httpd-2.0.55-win32-x86-src.zip!.

\item Download Apache binary installer
  \verb!apache_2.0.55-win32-x86-no_ssl.msi!.

\end{enumerate}

\subsubsection{Apache build}

\begin{enumerate}
\item Unpack \verb!httpd-2.0.55-win32-x86-src.zip! in \verb!C:\! creating
  \verb!C:\httpd-2.0.55! (keep zip file for now).

\item Open a shell and execute:
\begin{verbatim}
C:
CD \httpd-2.0.55
patch -p0 < C:\globule\apache\udp-requests-httpd-2.0.50.patch
\end{verbatim}
all hunks should succeed.

\item\label{step:buildwin} Put all the files from
  \verb!C:\globule\apache\GlobuleMonitor! into
  \verb!C:\httpd-2.0.55\support\win32!

  \textbf{Should be patched rather than overwritten.} \\
  \textbf{The user how always gets a patched ApacheMonitor}

\item Open Apache.dsw, convert dsp files to new Microsoft Visual Studio
  environment (answer ``Yes to all'').  Note that if you had already
  build Apache before and want to rebuild, do not open the Apache.dsw, but
  the Apache.vcproj file and naturally the next step can be skipped.
\item Unfortunately httpd-2.0.55 has been improperly released (that is: with
  SSL turned on), therefor you need to disable SSL using:
  
  \begin{itemize}
  \item select Apache project in Solution Explorer
  \item \texttt{Build} $\rightarrow$ \texttt{Configuration Manager};
    change \texttt{Active Solution Configuration} to ``Release''.
  \item do not let projects \texttt{abs}, \texttt{mod\_ssl}, and
    \texttt{mod\_deflate} be build (i.e. check them off).
  \end{itemize}
\item In \texttt{Solution Explorer} right click \texttt{ApacheMonitor} choose
  \texttt{Add} $\rightarrow$ \texttt{Add Existing Item};
  select \verb!GlobuleMonitor.c! and (using control key) also
  \verb!GlobuleMonitor.h! and select \texttt{Open} to add them.

\item In \texttt{Solution Explorer} right click \texttt{ApacheMonitor} select
  \texttt{Properties}.  Make sure the \texttt{Configuration} is the active
  configuration(Release).  Change the following:
  \begin{itemize}
  \item In \texttt{Configuration properties} $\rightarrow$ \texttt{Linker}
    $\rightarrow$ \texttt{General};  Add to \texttt{Additional Library
    Directories}
\begin{verbatim}
\verb!C:\httpd-2.0.55\srclib\apr\Release;C:\httpd-2.0.55\srclib\apr-iconv\Release;C:\httpd-2.0.55\srclib\apr-util\Release!
\end{verbatim}
  \item In \texttt{Configuration properties} $\rightarrow$ \texttt{Linker}
    $\rightarrow$ \texttt{Input};  Add to \texttt{Additional Dependencies} the
    following libraries:
\begin{verbatim}
comctl32.lib libapr.lib libapriconv.lib libaprutil.lib Ws2_32.lib
\end{verbatim}
  \item In \texttt{Configuration Properties} $\rightarrow$ \texttt{C/C++}
    $\rightarrow$ \texttt{General};  Add to \texttt{Additional Include
    Directories} the path:
\begin{verbatim}
C:\httpd-2.0.55\srclib\apr\include;C:\httpd-2.0.55\srclib\apr-util\include
\end{verbatim}
  \end{itemize}

\item Select the top-level project "Apache2" and build solution, Apache will
  be build to destination \verb!C:\Apache2!.
\item\label{step:endbuildwin} Remove all files named \verb!*.pdb! from the
  directories \verb!\C:\Apache2\bin!, \verb!C:\Apache2\bin\iconv!,
  \verb!C:\Apache2\lib!, and \verb!C:\Apache2\modules!
\item Rename \verb!C:\Apache2! to \verb!C:\ApacheInstallFiles\PatchedApache!.
\item Remove \verb!C:\httpd-2.0.55!.
\item Unpack \verb!httpd-2.0.55-win32-x86-src.zip! to \verb!C:\httpd-2.0.55!
  (zip may be discarded now if you really want to).
\item Repeat steps \ref{step:buildwin} thru \ref{step:endbuildwin}.

\end{enumerate}

\subsubsection{Globule build}

\begin{enumerate}
\item Open \verb!C:\globule\globule.sln! in Visual Studio.

\item \texttt{Build} $\rightarrow$ \texttt{Configuration Manager}
  $\rightarrow$ select ``Release'' build.
  Untag \texttt{mod\_psodium} and \texttt{auditor} if selected.
\item \texttt{Build} $\rightarrow$ \texttt{Build Solution}

\item \texttt{Build} $\rightarrow$ \texttt{Configuration Manager}
  $\rightarrow$ select ``Release DNS redirection'' build.
  Untag \texttt{mod\_psodium} and \texttt{auditor} if selected.
\item \texttt{Build} $\rightarrow$ \texttt{Rebuild Solution}

\item Copy \verb!C:\globule\mod_globule\Release\DNS\mod_globule.so! to
      \verb!C:\ApacheInstallFiles\PatchedApache\modules!.
\item Copy \verb!C:\globule\mod_globule\Release\mod_globule.so! to
      \verb!C:\ApacheInstallFiles\PreInstalled!.

\end{enumerate}

\subsubsection{Packaging}

\begin{enumerate}

\item Place \verb!apache_2.0.55-win32-x86-no_ssl.msi! place in
  \verb!C:\globule\winstall!.
\item Place files \verb!msvcp71.dll!, \verb!msvcr71.dll!, \verb!msvcp71d.dll!,
  and \verb!msvcr71d.dll! in \verb!C:\globule\winstall!.
\item Copy directory \verb!C:\globule\sample\globuleadm! to
      \verb!C:\globule\winstall!.  However do not copy the CVS subdirectory,
      there are also subdirectories which can contain CVS directories to be
      removed.
\item Copy directory \verb!C:\globule\sample\html! to
      \verb!C:\globule\winstall! and rename it to \verb!htdocs!.  However do
      not copy the CVS subdirectories, also not from any of the subdirectories.
\item Copy \verb\!C:\globule\sample\javascript-example.html! to
      \verb!C:\globule\winstall\htdocs\javascript-example.html!.
\item Copy and rename directory \verb!C:\globule\sample\errordocument! to
      \verb!C:\globule\winstall\error!.  Do not copy the CVS directory
      contained within it.
\item Copy both \verb!Globule.css! and \verb!globule.png! from
      \verb!C:\globule\winstall\globuleadm! to
      \verb!C:\globule\winstall\htdocs! and also both to
      \verb!C:\globule\winstall\htdocs\globule!.
\item Make sure the \verb!httpd-globule.conf! in the \verb!C:\globule\winstall!
      directory still reflects the converted \verb!globule\sample\httpd.conf!
      file.  Copy \verb!httpd-globule.conf! to \verb!httpd.conf!.
\item Copy \verb!ApacheMonitor.exe! from
      \verb!C:\ApacheInstallFiles\PatchedApache\bin! to
      \verb!C:\globule\winstall!

  \textbf{Problem that the ApacheMonitor already went into the regular
    package.}

\item Edit \verb!C:\globule\winstall\globule.nsh!
  \begin{itemize}
  \item check or modify the version number of supported Apache's
    \texttt{AP\_VER\_\textit{\{}MAJOR\textit{,}MINOR\textit{,}REVISION\_MIN\textit{,}REVISION\_MAX\textit{\}}}
  \item Check or modify the Apache window installer file version number
    at \texttt{AP\_INSTALL\_FILE}
    (currently \verb!apache\_2.0.55-win32-x85-no\_ssl.msi!).
  \item Check whether the utils defined at ``Globule main site'' are still
    valid.

  \end{itemize}

\item Right click \verb!C:\globule\winstall\globule.nsi! and select
  \texttt{Compile NSIS Script}.

\end{enumerate}

Target is left in \verb!C:\globule\winstall! as \texttt{Globule-1.3.1.exe}.

\textbf{Some files (a.o. \texttt{*.pdb}) should have been removed before
  packaging.}

\subsubsection{Notes on changing paths from above defaults}

There are a lot of hardcoded paths, some of them are:
\begin{itemize}
\item[-] class view $\rightarrow$ right mouse $\rightarrow$ mod\_globule $\rightarrow$ properties $\rightarrow$ all configurations
\item[-] set path in \verb!compat\win32\httpd.h! for real \verb!httpd.h!
\item set additional library directories
\end{itemize}

\subsubsection{Just running Globule}

\begin{itemize}
\item Copy \texttt{mod\_globule\\\textit{release-type}\\mod\_globule.so} to
modules directory of Apache, where the release-type is one of \texttt{Debug}, \texttt{Release}, \texttt{Debug\\DNS} or \texttt{Release\\DNS}
\end{itemize}

\subsection{Building the source based installer script}

\begin{itemize}
\item copy \verb!mod-globule-*.tar.gz! to
\verb!/home/ftp/pub/globule/auto/src/!

\item run \verb!./tools/mkinstaller.sh!

\end{itemize}

\subsection{Building the debian package}

\textbf{Missing.}

\subsection{Building the RPMs}

Start VMWare and perform the following actions in VMware:

\begin{enumerate}
\item start the shrike-src virtual machine
\item{\label{step:revertshrike} it should have been previously checkpointed to
   a running state so: revert to checkpoint}
\item a command should be up so resynchronize the time:
   !verb!ntpdate top.cs.vu.nl!,  execute this command.
\item reboot virtual machine cleanly (ie. give reboot command).
\item when it has been started, login as root with password ``geheim''.
\item The \verb!/usr/src/redhat! directory should have been mounted from the
   \verb!/usr/src/rpm! of the host-machine.  If not use measures to copy files
   back-and-forth and in steps running !verb!sync!.
\item \label{step:endrevertshrike} \verb!cd /usr/src/redhat/SPECS!.
\item \verb!rpm -i ../RPMS/i386/httpd-2.0.55.*!
\item \verb!rpmbuild -ba mod-globule-1.3.0.spec!
\item \verb!sync ; sleep 5!
\item repeat steps \ref{step:revertshrike} thru \ref{step:endrevertshrike}.
\item \verb!rpm -i ../RPMS/i386/httpd-local-2.0.55.*!
\item \verb!rpmbuild -ba mod-globule-local-1.3.0.spec!
\item \verb!sync ; sleep 5!
\item repeat steps \ref{step:revertshrike} thru \ref{step:endrevertshrike}.
\item \verb!rpm -i ../RPMS/i386/globule-1.3.0.spec!
\item \verb!sync ; sleep 5!
\item \verb!revert to snapshot and suspend virtual machine.!
\end{enumerate}

\subsection{Publication}

The source based installer script is immediately released, so the previous
section could best be done for the final release, not release candidates.

Copy the produced \verb!Globule-1.3.0.exe! file from the windows platform to
\verb!/home/ftp/pub/globule!.

Execute the \texttt{publish.sh} file to transport the produced
distribution files from the \verb!/usr/src/rpm! tree to
\verb!/home/ftp/pub/globule!.

\section{Release roll-out}

\textbf{Missing}

\subsection{Members.globule.org}

Adding members:
\begin{verbatim}
  ssh globe@mirror.cs.vu.nl
  su - groot
  vi /globule/members.globule.org/conf/httpd.conf
\end{verbatim}
Add the user to the \verb!UserDir enabled!\ldots line.
\begin{verbatim}
  vi /globule/htdocs/members/htdocs/index.html
\end{verbatim}
Add the user to the list of people using the Globule-replicated Home Pages.
\begin{verbatim}
  /globule/bin/globulectl restart
\end{verbatim}

\clearpage
\appendix
\section{\label{sec:servers}Servers in use}

\subsection{Per-machine listing}

Codes used:
\begin{description}
\item[Slackware] Host running Slackware, stand-alone.
\item[FC3] Host running RedHat Fedore Core 3
\item[RH/das] Host running RedHat, part of the das cluster, in machine room.
\item[\textit{xxx}/desktop] \ldots\/on someone's desk, being maintained by owner.
\item[\textit{xxx}/server] \ldots\/in the server room, maintained by Berry.
\end{description}

\begin{description}
\item[\bf\tt zappa.cs.vu.nl]
  \textit{\scriptsize RH/das, P3 700MHz 256MB 20GB}

  \begin{itemize}
  \item[-] demo.globule.org
  \item[-] redirector for members a.k.a. globule.cs.vu.nl.
  \end{itemize}

\item[\bf\tt baby.cs.vu.nl]
  \textit{\scriptsize RH/das, P3 700MHz 512MB 20GB}

  \begin{itemize}
  \item[-] main GBS/www.globeworld.net server.
  \end{itemize}

\item[\bf\tt posh.cs.vu.nl]
  \textit{\scriptsize RH/das, P3 550MHz 256MB 20GB}

  Currently out of the running --- stability issues.

\item[\bf\tt scary.cs.vu.nl]
  \textit{\scriptsize RH/das, P3 550MHz 256MB 9GB}

  \begin{itemize}
  \item[-] replica for www.globule.org.
  \end{itemize}

\item[\bf\tt wereld.cs.vu.nl]
  \textit{\scriptsize Slackware/server, Athlon 1.2GHz 512MB 40GB}

  \begin{itemize}
  \item[-] bugzilla
  \item[-] origin for www.globule.org.
  \end{itemize}

\item[\bf\tt goupil.cs.vu.nl]
  \textit{\scriptsize FC3/desktop, Athlon64 3000+ 2.0GHz 1024MB N/A}

  owner: Guillaume

\item[\bf\tt world.cs.vu.nl]
  \textit{\scriptsize Slackware/desktop, AthlonXP 2800+ 2.0GHz 1024MB 40+80GB}

  owner: Berry

\item[\bf\tt reich.cs.vu.nl]
  \textit{\scriptsize WindowsXPprof}

  Replica server for GBS.

\item[\bf\tt glass.cs.vu.nl]
  \textit{\scriptsize WindowsXPhome}

  Replica server for GBS.

\end{description}


Other machines (for reference purposes):
\begin{description}
\item[\bf\tt eendracht.cs.vu.nl] Linux router wide-area network emulator.
\item[\bf\tt *.inria.fr] Unused due to software problems (could perhaps be used again).
\item[\bf\tt globe.sdsc.edu] FreeBSD slave of globule.cs.vu.nl
\item[\bf\tt gsd.ime.usp.br] Linux slave of globule.cs.vu.nl
\item[\bf\tt firefly02.cs.cornell.edu] Linux slave of globule.cs.vu.nl
\item[\bf\tt mirror.cs.vu.nl] Solaris master of globule.cs.vu.nl
\item[\bf\tt sporty.cs.vu.nl] compute server.
\item[\bf\tt ginger.cs.vu.nl] compute server.
\end{description}

\newpage

\subsubsection{Updating SDSC (globe.sdsc.edu)}

On the SDSC located machine, everything is being compiled, installed and
running as user ``globe''.  The home directory \verb!! is the top-level
directory where Globule related files reside.

\begin{verbatim}
ssh globe@globe.sdsc.edu
cd build
wget 'http://www.globule.org/download/\globuleversion/mod-globule-\globuleversion.tar.gz'
tar xzf mod-globule-\globuleversion.tar.gz
cd mod-globule-\globuleversion
setenv CFLAGS -pthread
setenv CXXFLAGS -pthread
setenv CPPFLAGS -pthread
setenv SHELL /usr/local/bin/bash
./configure --with-apache=/home/globe/globule/members.globule.org/
make
make install
cd ../../globule/members.globule.org
vi conf/httpd.conf ;# optionally change settings
cd bin
./our-apachectl start
\end{verbatim}

\subsubsection{Updating Cornell}

On the Cornell located machine, everything is being compiled, installed and
running as user ``globe''.  The home directory \verb!/home/globe/globule!
is the top-level directory where Globule related files reside.  The installer
script is used to update Cornell.

\begin{verbatim}
ssh globe@firefly02.cs.cornell.edu
/home/globe/globule/src/globule.sh
\end{verbatim}

The \verb!globule.sh! is a simple front end for the installer script with
standard arguments.

\subsubsection{Updating Brazil}

On the site in Brazil, everything is being compiled, installed and running as user ``globe''.  The home directory \verb!/home/globe! is the top-level directory where Globule related files reside.  The directory \verb!build! is used to compile the sources, installation happens in \verb!!.
ssh globe@

\subsubsubsection*{Optionally upgrading Apache}

\begin{verbatim}
ssh globe@gsd.ime.usp.br
cd build
wget 'http://www.globule.org/download/apache-2.0.55/httpd-2.0.55.tar.gz'
tar xzf httpd-2.0.55.tar.gz
cd httpd-2.0.55
./configure --enable-so --disable-cgi --disable-cgid --disable-proxy --prefix=/home/globe/globule/members.globule.org
make
make install
cd ../../globule/members.globule.org/
vi conf/httpd.conf ;# optionally change settings
/usr/u/globe/globule/members.globule.org/bin/apachectl start
\end{verbatim}

\begin{verbatim}
ssh globe@gsd.ime.usp.br
cd build
wget 'http://www.globule.org/download/1.3.0/mod-globule-1.3.0.tar.gz'
tar xzf mod-globule-1.3.0.tar.gz
cd mod-globule-1.3.0
./configure --with-apache=/home/globe/globule/members.globule.org/
make
make install
cd /home/globe/globule/members.globule.org
vi conf/httpd.conf ;# Optionally update settings
cd bin
./our-apachectl start
\end{verbatim}

\subsection{Role based listing}

\begin{description}
\item[\bf\tt www.globule.org]

\begin{tabular}
\texttt{wereld.cs.vu.nl} & Origin & Slackware machine in serverroom \\
\texttt{scary.cs.vu.nl}  & Slave  & \\
\end{tabular}

\item[\bf\tt globule.cs.vu.nl]

Members the set of servers where the user private web pages of the
\texttt{cs.vu.nl} can be replicated.  If users in the \texttt{.cs.vu.nl}
domain put the following line
{\scriptsize \begin{verbatim}
<meta http-equiv="Refresh" content="0;url=http://members.globule.org:8041/~user/http-index.html"/>
\end{verbatim}
replacing \verb!user! with their username, in their
\texttt{$HOME/www/index.html} then browsing users will be redirected
to members.globule.org:8041 for which DNS redirection is used to forward the
browser to one of the replica servers.
The DNS redirector is a stand-alone redirector, while the origin is running on
a Sun machine because it needs a secure NFS key to access the home directories
of the users.  The redirector needs to be seperate in order to be accessible
on port 53.  Because of the DNS redirection, and the replica servers cannot
(all) access port 80 (the remote server Apaches are running as regular users)
all systems need to run Apache on port 8041.

\begin{tabular}
\texttt{mirror.cs.vu.nl} & Origin     & Sun in machine room to access users home directory \\
\texttt{zappa.cs.vu.nl}  & Redirector & Stand-alone redirector for members.globule.org \\
\texttt{firefly02.cs.cornell.edu} & Replica & Slave replica server \\
\texttt{globe.sdsc.edu}  & Replica    & Slave replica server --- out of commission \\
\texttt{gsd.ime.usp.br}  & Replica    & Slave replica server --- out of commission \\
\end{tabular}

\section{\label{sec:vmware}VMware images}

\textbf{VMware images should be available on a physical image}.

On the Unix/Linux based distribution of the VMware images, solely the packages
are build under the superuser credentials.  All images use password ``geheim''
for the root account.

\begin{description}
\item{shrike-src}
\item{debian-testing}
\item{winxpdev}
\end{description}

\section{\label{sec:prerequisites}Misceleneous software}

\begin{description}
\item{patch for windows}
\item{awk for windows}
Available from \verb!http://cm.bell-labs.com/cm/cs/who/bwk/! as
\verb!http://cm.bell-labs.com/cm/cs/who/bwk/awk95.exe! but rename to
\verb!C:\windows\awk.exe!.

\textbf{awk.exe and patch.exe should be included in the images cd.} \\
\textbf{msvcp71.dll, msvcr71.dll, msvcp71d.dll, and msvcr71d.dll should
also be included}

\end{description}

\section{\label{sec:history}Historic log of changes}

For documentation purposes, important changes (not additions) in the packaging
and distribution are described here.  This in order to keep track of changes
make in the CVS repository and that the right procedure is kept for historic
releases.

\begin{itemize}
\item Note 2005-07-29: The old server \verb!ensemble02.cs.cornell.edu! has been
fased out.  The content of the home directory as it was on ensemble02 is
stored in the subdirectory \verb!/home/globe/ensemble!.
\end{itemize}

\end{document}
